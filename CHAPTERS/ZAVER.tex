\chapter*{Závěr}
\addcontentsline{toc}{chapter}{Závěr}
Záměrem mojí práce bylo vytvořit univerzální systém pro automatizaci skle\-ní\-ku, který je:
\begin{itemize}
    \item open-source
    \item levný
    \item modulární
    \item snadný na ovládání
    \item univerzální
\end{itemize}

Tento cíl se mi podařilo splnit.
Lidé, kteří mají zájem si systém vytvořit najdou veškerou dokumentaci, schémata a~zdrojový kód na webu \textit{\url{www.protoplant.cz}}.

Díky SOČ jsem se naučil pracovat se softwarem pro návrh PCB Autodesk EAGLE.
Zároveň jsem vylepšil své schopnosti v~programování a~získal spoustu dalších zkušeností v~elektrotechnice a~s~prací na takto komplexních projektech.

Část svých plánů do budoucna jsem již nastínil v~kapitole \ref{sec:DISTRIBUTION}.
PROTOPlant plánuji začít vyrábět průmyslově a~prodávat jej.
Zároveň jej budu dále vylepšovat a~přidávat další funkce.
V~bližší době plánuji dokončit všechny právě rozpracované moduly a~začít implementovat funkci pro vzdálený přístup a~sledování přes internet (například z~práce, nebo z~dovolené).
Dále bych pro PROTOPlant vyvinul vlastní mobilní aplikaci, která uživateli umožní snadno a~rychle vzdáleně sledovat stav skleníku, měnit nastavení, nebo jej ovládat.

Momentálně běží PROTOPlant v~jednom skleníku. 
Toto číslo bych během následujícího roku rád alespoň zdvacetinásobil. 

\newpage